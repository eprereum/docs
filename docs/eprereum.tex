\documentclass[twocolumn]{article}
\usepackage{amsmath}
\usepackage{tipa}

\begin{document}

\title{EPR-ereum \\ bridging different worlds}
\author{Benjamin Bollen}
\maketitle
%\abstract{}
\section*{Overview}
Proof-of-Work mining successfully secures Ethereum, however it does not allow the computational power of Ethereum to scale as more full nodes join the network.

We present an overlay protocol to scale the capacity of decentralised applications on public Ethereum. Our solution manages all value on Ethereum and provides an on-chain marketplace for pooling computational resources of full nodes allowing them to contribute and earn gas rewards linearly from the execution power they provide to Ethereum.

The solution does not modify the scaling properties of Ethereum directly, rather it aims to get more computational \emph{gas mileage} out of the same Ether by minimising the verification burden put on the Proof-of-Work miners.  To this end we introduce a Byzantine fault-tolerant, staked meta-consensus mechanism. This mechanism allows us to scale computational capacity with the number of groups of nodes available.  By moving the verification burden away from the Proof-of-Work miners we bypass the \emph{verifiers dilemma}\footnote{The verifiers dilemma states that the verification burden for honest miners in Nakamoto consensus systems must be small compared to the sealing effort lest the honest miners be vulnerable to attacks. We will discuss the dilemma in more detail further.} while retaining the full replay history of Ethereum.

We detail the protocol and discuss an early analysis of attack vectors. We imagine possible use-cases and review the gains by enabling Eprereum for the application. At the end we place this proposal in relation to existing work on scaling Ethereum.
%We discuss how our proposal is complementary to the work done on Proof-of-Stake Casper and Truebit; and how it falls in a different category than Cosmos and Polkadot based on the different problem it solves for. % We highlight this is not a side channel approach, as the objective is to explicitly keep all valuable state on Ethereum after every transaction completes


\section{It's all about gas}
% describe gas and relation to Ether. (pick up later for EPR token for organisation
% present the problem: proof-of-work implies verifiers dilemma;
% cost of transaction execution high because verifiers dilemma
% Tendermint chains have known validators (counter to POW) so can share gas rewards; no race.
% Implement EPRBlockGasLimit.

In Ethereum gas accounts for every execution step.  A transaction includes a \emph{gas price} and a \emph{gas limit}.  The gas limit sets the maximum gas that can be burnt during the execution of this transaction.  When the execution completes the total gas used is deducted from the account balance of the transaction \emph{sender} at the set gas price\footnote{The gas price is set in Ether per gas [ETH], where gas is a number.}.

In late spring of 2017 a steep increase in the number of transactions processed on the Ethereum blockchain has been registered.  All the while the average gas used per transaction has remained in first approximation constant. The accompanying higher market valuation of Ether had the price per transaction sky-rocket. In response the average gas price has started a correction downwards at the time of this writing. [graph]

This demonstrates that the Ethereum protocol has scaled successfully under an increase of the number of transactions. However, it leaves unanswered how Ethereum can scale for more computationally intensive applications.  We will argue that it is not only the price that is prohibitive to build applications that consume more gas. It has been highlighted that Nakamoto consensus systems have an inherent requirement to keep the block validation to a mimimum.  Consequently this keeps the gas price for transactions validated by the Proof-of-Work miners high, as the total block fee, $f = \sum_{tx} \text{gas} \cdot \text{gasprice}$.  From this it is clear that for honest miners security and profitability are optimised with a supply of many low-gas, high-gas price transactions.

The security that Proof-of-Work miners generate is rooted in the adversarial race the miners are in to solve the block sealing puzzle first.  Only the miner who contributes the block is awarded the block rewards and block fee.  Other miners are expected to validate the correctness of the transactions in a block for the good of the network without reward.  They obviously have an incentive to make sure the block does not contain invalid transactions, because that would invalidate all work they build on top of it.  Verifying transactions of mined blocks, however, carries the risk of falling behind on mining the new block.  This is a brief summary of what has been understood as the verifiers dilemma\footnote{see appendix for more details (todo)}[Teutsch].

If the verifiers dilemma is implied by the Nakamoto consensus and its reward structure, then it can be avoided by examining other incentive structures.  The mining paradigm to date forms the foundation of the two biggest permissionless blockchains %\footnote{in hashing power, or market capitalisation, or ...} 
and as such our objective in this work is not to replace it.  Rather we will present an opt-in, overlay protocol that constructs a new scaling dimension for decentralised applications on Ethereum.


\section{Bridging different worlds}

Going forward we define the results of the Proof-of-Work consensus of Ethereum to be true.  We note that a Nakamoto consensus engine can roll back state, however, our construction will be local to a block and as such invariant to which branch of the consensus engine eventually accumulates most work.

As an overlay protocol on Ethereum, all Eprereum nodes are constructed to verify the Ethereum blockchain.  As a result, we can use smart contracts on Ethereum to construct a deterministic, global\footnote{Note that this global symmetry is only valid under the stated assumption that the Eprereum network is invariant under the probabilistic finalisation of Ethereum.  Here without loss of generality we assume that the global symmetry is valid, but this is a requirement for an implementation to observe.} one-to-many communication channel without incurring an overhead on the number of Eprereum nodes in the network.

Communication in the other direction (from Eprereum nodes back to Ethereum), however, is expensive as it requires transactions to be validated by the Proof-of-Work miners at a high gas price.

A transaction $t$ on Ethereum transforms the full state $S$ to a new state $S'$.  The Ethereum Virtual Machine (EVM) is explicitly constructed to be serial in its execution, so we can compose transactions:
\begin{align}
	t_1 \circ t_2&: S \rightarrow S'' \\
	\text{where } t_1&: S \rightarrow S',
	t_2: S' \rightarrow S''. \nonumber
\end{align}
For a transaction that calls a constant function at the end of its execution we can note:
\begin{align}
	t = \tilde{t} \circ \tilde{t}_c&: S \rightarrow S', r \\
	\text{where } \tilde{t}&: S \rightarrow S', \tilde{t}_c: S' \rightarrow S', r \nonumber
\end{align}
with $r$ the result returned from the function call $\tilde{t}_c$.

While the execution of a single transaction is atomic and serialised, smart contract developers need to code smart contracts with the understanding that the order of transactions is not under their control as it is determined by the block formation of Proof-of-Work miners.  We therefore introduce an \textit{asynchronous callback} $\cdot$ composition we can use to construct a composed transaction:
\begin{align}
	t = \tilde{t}_0 \left(\prod_{i=1}^{N} \circ \ \tilde{t}_i \circ \tilde{t}_{c, i} \right) \circ \tilde{t}_{N+1} \circ \left(\prod_{i=1}^N \cdot \ \tilde{t}_{r, i}\right),
\end{align}
where $\tilde{t}_{c, i}$ call constant functions and $\tilde{t}_{r, i}$ call (non-constant) functions with the asynchronously returned result $r_i$.  Where we require that $\circ \ ( t_{r, 1} \cdot t_{r, 2} )$ must equal $\circ \ ( t_{r, 2} \cdot t_{r, 1} )$, as the result will be returned through the Nakamoto consensus algorithm, and no guarantee on the order can be given.  Note that this not more difficult than standard asynchronous programming as the program has no control over the scheduler that orders the asynchronous callbacks.

Without loss of generality we will further consider transactions of the form
\begin{align}
	t = \tilde{t} \circ \tilde{t}_c \cdot \tilde{t}_r,
\end{align}
where $\tilde{t}$ and $\tilde{t}_r$ write to the Ethereum state, and $\tilde{t}_c$ only reads from the Ethereum state, returning a single result $r$.
% argue: minimal change, reuse tested logic in virtual level where possible.
% present top-view of solution: virtual BFT meta-consensus mechanism
% nodes in this meta consensus are in themselves BFT Eprereum chains consistent of Eprereum nodes
\section{Connecting the wires}
% On ETH memory-pool contract to process results
% GasLimit; charged used gas or 80% of GasLimit which ever one is higher
\section{Steering network health}

% ------- DRAFT

%The core of this proposal is to realise that the consensus algorithm of a blockchain effectively defines the \textit{laws of nature} for the application running on top. For Proof-of-Work consensus this implies the Verifiers Dilemma that keeps the gas price in the case of Ethereum high; for a Tendermint consensus-based chain this implication does not follow. We aim to exploit that to strengthen the execution power of public Ethereum.  The over-stretched analogy to physics here is that a \emph{portal} contract on public Ethereum can function as an escrow-protected callback function to a mirror-chain of the public Ethereum. This mirror-chain functions as parallel universe, with different laws of nature, in particular in this chain the calling contract can bypass the Verifiers Dilemma it encounters on the Ethereum chain.  Results from execution can be sent back through

%\section*{Pre-ramble}
%
%First off, this is not a grandstanding exercise to claim any major solutions; nor is it a research paper that would hold an academic standard.  It probably should be a blog-post, but up to today the author has failed to utilise his Medium account.  Additionally most of the merit of the proposal here goes to many people who keep working hard to advance the state-of-the-art in decentralisation technology, and the author will attempt to acknowledge them duly at the end.  That said, the author hopes that despite the informal tone and self-deprecating humour, the readers will find value in the ideas presented.
%
%EPR-ereum, \textipa{/"@prIr:@m/}\footnote{The author does not know how to write IPA, but it looks cute.}, takes the name from the Einstein-Podolsky-Rosen (EPR) paradox, on particle entanglement and a more recent conjecture in physics, \mbox{$\text{ER}=\text{EPR}$}, stating an equivalence between wormholes and such entangled particles to capture the core idea of this proposal, but really, it's a cheap shot at finding a name, and at best a mnemonic trick for physicists\footnote{\label{corefootnote}The core of this proposal is to realise that the consensus algorithm of a blockchain effectively defines the \textit{laws of nature} for the application running on top. For Proof-of-Work consensus this implies the Verifiers Dilemma that keeps the gas price in the case of Ethereum high; for a Tendermint consensus-based chain this implication does not follow. We aim to exploit that to strengthen the execution power of public Ethereum.  The over-stretched analogy to physics here is that a \emph{portal} contract on public Ethereum can function as an escrow-protected callback function to a mirror-chain of the public Ethereum. This mirror-chain functions as parallel universe, with different laws of nature, in particular in this chain the calling contract can bypass the Verifiers Dilemma it encounters on the Ethereum chain.  Results from execution can be sent back through }.  There is no argument to actual physics here. It's worth noting that the rest of the article is an expansion of that footnote \ref{corefootnote}; so you can read that and call it a day.
%
%Lastly, the author hopes that the comical tone will help the readers assert that this not intended to be (yet another) \emph{Minimum Viable Whitepaper}.  Rather the advent of the token economy puts additional pressure on the infrastructure in this space, and raises the need to scale the capacity of the backbone faster.  This is the context
%- market for execution price
%- parallel chain; full mirror of Ethereum state, but through portal contract call into parallel chain where gasprice is as agreed in Ethereum 

\end{document}
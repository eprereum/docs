\documentclass[twocolumn]{article}
\usepackage{amsmath}
\usepackage{tipa}

\begin{document}

\title{EPR-ereum \\ bridging different worlds}
\author{Benjamin Bollen}
\maketitle
\abstract{We present a pragmatic proposal to scale the capacity of decentralised applications on public Ethereum. We also discuss how this proposal is complementary to the work done on Proof-of-Stake Casper and Truebit; and how it falls in a different category than Cosmos and Polkadot based on the different problem it is trying to solve for.
%The presentation builds on theFurthermore, we try to structure the marketplacedecentralised 

This is }

\section*{Pre-ramble}

First off, this is not a grandstanding exercise to claim any major solutions; nor is it a research paper that would hold an academic standard.  It probably should be a blog-post, but up to today the author has failed to utilise his Medium account.  Additionally most of the merit of the proposal here goes to many people who keep working hard to advance the state-of-the-art in decentralisation technology, and the author will attempt to acknowledge them duly at the end.  That said, the author hopes that despite the informal tone and self-deprecating humour, the readers will find value in the ideas presented.

EPR-ereum, \textipa{/"@prIr:@m/}\footnote{The author does not know how to write IPA, but it looks cute.}, takes the name from the Einstein-Podolsky-Rosen (EPR) paradox, on particle entanglement and a more recent conjecture in physics, \mbox{$\text{ER}=\text{EPR}$}, stating an equivalence between wormholes and such entangled particles to capture the core idea of this proposal, but really, it's a cheap shot at finding a name, and at best a mnemonic trick for physicists\footnote{\label{corefootnote}The core of this proposal is to realise that the consensus algorithm of a blockchain effectively defines the \textit{laws of nature} for the application running on top. For Proof-of-Work consensus this implies the Verifiers Dilemma that keeps the gas price in the case of Ethereum high; for a Tendermint consensus-based chain this implication does not follow. We aim to exploit that to strengthen the execution power of public Ethereum.  The over-stretched analogy to physics here is that a \emph{portal} contract on public Ethereum can function as an escrow-protected callback function to a mirror-chain of the public Ethereum. This mirror-chain functions as parallel universe, with different laws of nature, in particular in this chain the calling contract can bypass the Verifiers Dilemma it encounters on the Ethereum chain.  Results from execution can be sent back through }.  There is no argument to actual physics here and the rest of the paper is effectively an expansion of footnote \ref{corefootnote}.
%- market for execution price
%- parallel chain; full mirror of Ethereum state, but through portal contract call into parallel chain where gasprice is as agreed in Ethereum 

\end{document}